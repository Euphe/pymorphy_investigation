\documentclass[a4paper,14pt]{article}

%\includeonly{topics/t2,topics/t3} % компилировать только указанные главы
%%% Работа с русским языком
\usepackage{cmap}					% поиск в PDF
\usepackage{mathtext} 				% русские буквы в формулах
\usepackage[T2A]{fontenc}			% кодировка
\usepackage[utf8]{inputenc}			% кодировка исходного текста
\usepackage[english,russian]{babel}	% локализация и переносы

\usepackage{polyglossia}
\setmainlanguage{russian} 
\setotherlanguage{english}

% XeLaTeX can use any font installed in your system fonts folder
% Linux Libertine in the next line can be replaced with any 
% OpenType or TrueType font that supports the Cyrillic script.

\newfontfamily\russianfont[Script=Cyrillic]{Linux Libertine}

\frenchspacing
%\usepackage{fontspec} 
%\setmainfont[Ligatures={TeX,Historic}]{Times New Roman}

%%%\usepackage{fancyhdr} % {\tiny  }Колонтитулы 
%\pagestyle{fancy} 
%\renewcommand{\headrulewidth}{0mm} % Толщина линейки, отчеркивающей верхний колонтитул 
%\lfoot{Нижний левый} 
%\rfoot{Нижний правый} 
%\rhead{Верхний правый} 
%\chead{Верхний в центре} 
%\lhead{Верхний левый} 
% \cfoot{Нижний в центре} % По умолчанию здесь номер страницы 

%%% Дополнительная работа с математикой
\usepackage{amsmath,amsfonts,amssymb,amsthm,mathtools} % AMS
\usepackage{icomma} % "Умная" запятая: $0,2$ --- число, $0, 2$ --- перечисление
%\usepackage{ dsfont } % more math fonts!
%\renewcommand{\epsilon}{\ensuremath{\varepsilon}}
%\renewcommand{\phi}{\ensuremath{\varphi}}
%\renewcommand{\kappa}{\ensuremath{\varkappa}}
%\renewcommand{\le}{\ensuremath{\leqslant}}
\renewcommand{\leq}{\ensuremath{\leqslant}}
%\renewcommand{\ge}{\ensuremath{\geqslant}}
\renewcommand{\geq}{\ensuremath{\geqslant}}
%\renewcommand{\emptyset}{\varnothing}

%% Номера формул
%\mathtoolsset{showonlyrefs=true} % Показывать номера только у тех формул, на которые есть \eqref{} в тексте.
%\usepackage{leqno} % Нумереация формул слева

%%% Работа с картинками
\usepackage{graphicx}  % Для вставки рисунков
\graphicspath{crimeimages/}  % папки с картинками
\setlength\fboxsep{3pt} % Отступ рамки \fbox{} от рисунка
\setlength\fboxrule{1pt} % Толщина линий рамки \fbox{}
%\usepackage{wrapfig} % Обтекание рисунков текстом

\usepackage{indentfirst}






%%% Работа с таблицами
\usepackage{array,tabularx,tabulary,booktabs} % Дополнительная работа с таблицами
\usepackage{longtable}  % Длинные таблицы
\usepackage{multirow} % Слияние строк в таблице

%% Свои команды
\DeclareMathOperator{\sgn}{\mathop{sgn}}

%% Перенос знаков в формулах (по Львовскому)
\newcommand*{\hm}[1]{#1\nobreak\discretionary{}
	{\hbox{$\mathsurround=0pt #1$}}{}}

\newcommand{\source}[1]{\small{Источник: #1}}

%%% Страница
\usepackage{extsizes} % Возможность сделать 14-й шрифт
\usepackage{geometry} % Простой способ задавать поля
\geometry{top=20mm}
\geometry{bottom=20mm}
\geometry{left=35mm}
\geometry{right=15mm}
%
%\usepackage{fancyhdr} % Колонтитулы
% 	\pagestyle{fancy}
%\renewcommand{\headrulewidth}{0pt}  % Толщина линейки, отчеркивающей верхний колонтитул
% 	\lfoot{Нижний левый}
% 	\rfoot{Нижний правый}
% 	\rhead{Верхний правый}
% 	\chead{Верхний в центре}
% 	\lhead{Верхний левый}
%	\cfoot{Нижний в центре} % По умолчанию здесь номер страницы

\usepackage{setspace} % Интерлиньяж
\onehalfspacing % Интерлиньяж 1.5
%\doublespacing % Интерлиньяж 2
%\singlespacing % Интерлиньяж 1

\usepackage{lastpage} % Узнать, сколько всего страниц в документе.
%\usepackage{soul} % Модификаторы начертания

\usepackage{hyperref}
\usepackage[usenames,dvipsnames,svgnames,table,rgb]{xcolor}
\hypersetup{				% Гиперссылки
	unicode=true,           % русские буквы в раздела PDF
	%	pdftitle={Заголовок},   % Заголовок
	%	pdfauthor={Автор},      % Автор
	%	pdfsubject={Тема},      % Тема
	%	pdfcreator={Создатель}, % Создатель
	%	pdfproducer={Производитель}, % Производитель
	%	pdfkeywords={keyword1} {key2} {key3}, % Ключевые слова
	colorlinks=true,       	% false: ссылки в рамках; true: цветные ссылки
	linkcolor=blue,          % внутренние ссылки
	citecolor=blue,        % на библиографию
	filecolor=magenta,      % на файлы
	urlcolor=ForestGreen           % на URL
}


% \usepackage{csquotes} % Еще инструменты для ссылок
%\usepackage[style=authoryear,maxcitenames=2,backend=biber,sorting=nty]{biblatex}
\usepackage{multicol} % Несколько колонок
\usepackage{listings}
%%% Программирование
\usepackage{etoolbox} % логические операторы
%\usepackage[style=authoryear,maxcitenames=2,backend=biber,sorting=nty]{biblatex}

\usepackage{tikz} % Работа с графикой
% \usepackage{pgfplots}
% \usepackage{pgfplotstable}

% % % Для работы с report
%\renewcommand{\chaptername}{Глава}


\usepackage{cite} % Работа с библиографией
%\usepackage[superscript]{cite} % Ссылки в верхних индексах
%\usepackage[nocompress]{cite} % 
%\usepackage{csquotes} % Еще инструменты для ссылок

% \usepackage{natbib} %библиография плюс https://ru.wikibooks.org/wiki/LaTeX/%D0%A3%D0%BF%D1%80%D0%B0%D0%B2%D0%BB%D0%B5%D0%BD%D0%B8%D0%B5_%D0%B1%D0%B8%D0%B1%D0%BB%D0%B8%D0%BE%D0%B3%D1%80%D0%B0%D1%84%D0%B8%D0%B5%D0%B9
%\bibliographystyle{rmpaps} 


%%%% Работа со списками
%\usepackage{enumitem} %%% дополнительная работа со списками. !а может и неправильно написано — гугли
\usepackage{paralist}										% compact lists

\renewcommand{\citeleft}{(}
\renewcommand{\citeright}{)}

\usepackage{adjustbox} 
\addto\captionsrussian{% Replace "english" with the language you use
	\renewcommand{\contentsname}%
	{Оглавление}%
}

\author{Борис Цейтлин}
\title{Computational Linguistics and Text Analysis hw1}

   % We will generate all images so they have a width \maxwidth. This means
    % that they will get their normal width if they fit onto the page, but
    % are scaled down if they would overflow the margins.
    \makeatletter
    \def\maxwidth{\ifdim\Gin@nat@width>\linewidth\linewidth
    \else\Gin@nat@width\fi}
    \makeatother
    \let\Oldincludegraphics\includegraphics
    % Set max figure width to be 80% of text width, for now hardcoded.
    \renewcommand{\includegraphics}[1]{\Oldincludegraphics[width=.8\maxwidth]{#1}}
    % Ensure that by default, figures have no caption (until we provide a
    % proper Figure object with a Caption API and a way to capture that
    % in the conversion process - todo).
    \usepackage{caption}
    \DeclareCaptionLabelFormat{nolabel}{}
    \captionsetup{labelformat=nolabel}

    \usepackage{adjustbox} % Used to constrain images to a maximum size 
    \usepackage{xcolor} % Allow colors to be defined
    \usepackage{enumerate} % Needed for markdown enumerations to work
    \usepackage{geometry} % Used to adjust the document margins
    \usepackage{amsmath} % Equations
    \usepackage{amssymb} % Equations
    \usepackage{textcomp} % defines textquotesingle
    % Hack from http://tex.stackexchange.com/a/47451/13684:
    \AtBeginDocument{%
        \def\PYZsq{\textquotesingle}% Upright quotes in Pygmentized code
    }
  %  \usepackage{upquote} % Upright quotes for verbatim code
%    \usepackage{eurosym} % defines \euro
%    \usepackage[mathletters]{ucs} % Extended unicode (utf-8) support
%    \usepackage[utf8x]{inputenc} % Allow utf-8 characters in the tex document
    \usepackage{fancyvrb} % verbatim replacement that allows latex
    \usepackage{grffile} % extends the file name processing of package graphics 
                         % to support a larger range 
    % The hyperref package gives us a pdf with properly built
    % internal navigation ('pdf bookmarks' for the table of contents,
    % internal cross-reference links, web links for URLs, etc.)
    \usepackage{hyperref}
   \usepackage{longtable} % longtable support required by pandoc >1.10
   \usepackage{booktabs}  % table support for pandoc > 1.12.2
    \usepackage[inline]{enumitem} % IRkernel/repr support (it uses the enumerate* environment)
    \usepackage[normalem]{ulem} % ulem is needed to support strikethroughs (\sout)
                                % normalem makes italics be italics, not underlines
    

\definecolor{dkgreen}{rgb}{0,0.6,0}
\definecolor{gray}{rgb}{0.5,0.5,0.5}
\definecolor{mauve}{rgb}{0.58,0,0.82}
\definecolor{listinggray}{gray}{0.5}
\definecolor{lbcolor}{rgb}{1,1,1}

\setmonofont{Ubuntu Mono}
\renewcommand*\lstlistingname{Listing}

\lstset{
  % backgroundcolor=\color{lbcolor},
  tabsize=4,
  rulecolor=,
  language=Python,
  literate={<-}{\texttt{<-}}{1},
  basicstyle=\scriptsize\ttfamily,
  upquote=true,
  aboveskip={1\baselineskip},
  columns=fixed,
  showstringspaces=false,
  extendedchars=true,
  breaklines=true,
  mathescape=false,
  prebreak=\raisebox{0ex}[0ex][0ex]{\ensuremath{\hookleftarrow}},
  frame=tb,
  showtabs=false,
  showspaces=false,
  showstringspaces=false,
  keywordstyle=\color[rgb]{0, 0, 1},
  commentstyle=\color[rgb]{0.133, 0.545, 0.133},
  stringstyle=\color[rgb]{0.627, 0.126, 0.941},
  alsoletter={.}
}




    
    
    % Colors for the hyperref package
    \definecolor{urlcolor}{rgb}{0,.145,.698}
    \definecolor{linkcolor}{rgb}{.71,0.21,0.01}
    \definecolor{citecolor}{rgb}{.12,.54,.11}

    % ANSI colors
    \definecolor{ansi-black}{HTML}{3E424D}
    \definecolor{ansi-black-intense}{HTML}{282C36}
    \definecolor{ansi-red}{HTML}{E75C58}
    \definecolor{ansi-red-intense}{HTML}{B22B31}
    \definecolor{ansi-green}{HTML}{00A250}
    \definecolor{ansi-green-intense}{HTML}{007427}
    \definecolor{ansi-yellow}{HTML}{DDB62B}
    \definecolor{ansi-yellow-intense}{HTML}{B27D12}
    \definecolor{ansi-blue}{HTML}{208FFB}
    \definecolor{ansi-blue-intense}{HTML}{0065CA}
    \definecolor{ansi-magenta}{HTML}{D160C4}
    \definecolor{ansi-magenta-intense}{HTML}{A03196}
    \definecolor{ansi-cyan}{HTML}{60C6C8}
    \definecolor{ansi-cyan-intense}{HTML}{258F8F}
    \definecolor{ansi-white}{HTML}{C5C1B4}
    \definecolor{ansi-white-intense}{HTML}{A1A6B2}

    % commands and environments needed by pandoc snippets
    % extracted from the output of `pandoc -s`
    \providecommand{\tightlist}{%
      \setlength{\itemsep}{0pt}\setlength{\parskip}{0pt}}
    \DefineVerbatimEnvironment{Highlighting}{Verbatim}{commandchars=\\\{\}}
    % Add ',fontsize=\small' for more characters per line
    \newenvironment{Shaded}{}{}
    \newcommand{\KeywordTok}[1]{\textcolor[rgb]{0.00,0.44,0.13}{\textbf{{#1}}}}
    \newcommand{\DataTypeTok}[1]{\textcolor[rgb]{0.56,0.13,0.00}{{#1}}}
    \newcommand{\DecValTok}[1]{\textcolor[rgb]{0.25,0.63,0.44}{{#1}}}
    \newcommand{\BaseNTok}[1]{\textcolor[rgb]{0.25,0.63,0.44}{{#1}}}
    \newcommand{\FloatTok}[1]{\textcolor[rgb]{0.25,0.63,0.44}{{#1}}}
    \newcommand{\CharTok}[1]{\textcolor[rgb]{0.25,0.44,0.63}{{#1}}}
    \newcommand{\StringTok}[1]{\textcolor[rgb]{0.25,0.44,0.63}{{#1}}}
    \newcommand{\CommentTok}[1]{\textcolor[rgb]{0.38,0.63,0.69}{\textit{{#1}}}}
    \newcommand{\OtherTok}[1]{\textcolor[rgb]{0.00,0.44,0.13}{{#1}}}
    \newcommand{\AlertTok}[1]{\textcolor[rgb]{1.00,0.00,0.00}{\textbf{{#1}}}}
    \newcommand{\FunctionTok}[1]{\textcolor[rgb]{0.02,0.16,0.49}{{#1}}}
    \newcommand{\RegionMarkerTok}[1]{{#1}}
    \newcommand{\ErrorTok}[1]{\textcolor[rgb]{1.00,0.00,0.00}{\textbf{{#1}}}}
    \newcommand{\NormalTok}[1]{{#1}}
    
    % Additional commands for more recent versions of Pandoc
    \newcommand{\ConstantTok}[1]{\textcolor[rgb]{0.53,0.00,0.00}{{#1}}}
    \newcommand{\SpecialCharTok}[1]{\textcolor[rgb]{0.25,0.44,0.63}{{#1}}}
    \newcommand{\VerbatimStringTok}[1]{\textcolor[rgb]{0.25,0.44,0.63}{{#1}}}
    \newcommand{\SpecialStringTok}[1]{\textcolor[rgb]{0.73,0.40,0.53}{{#1}}}
    \newcommand{\ImportTok}[1]{{#1}}
    \newcommand{\DocumentationTok}[1]{\textcolor[rgb]{0.73,0.13,0.13}{\textit{{#1}}}}
    \newcommand{\AnnotationTok}[1]{\textcolor[rgb]{0.38,0.63,0.69}{\textbf{\textit{{#1}}}}}
    \newcommand{\CommentVarTok}[1]{\textcolor[rgb]{0.38,0.63,0.69}{\textbf{\textit{{#1}}}}}
    \newcommand{\VariableTok}[1]{\textcolor[rgb]{0.10,0.09,0.49}{{#1}}}
    \newcommand{\ControlFlowTok}[1]{\textcolor[rgb]{0.00,0.44,0.13}{\textbf{{#1}}}}
    \newcommand{\OperatorTok}[1]{\textcolor[rgb]{0.40,0.40,0.40}{{#1}}}
    \newcommand{\BuiltInTok}[1]{{#1}}
    \newcommand{\ExtensionTok}[1]{{#1}}
    \newcommand{\PreprocessorTok}[1]{\textcolor[rgb]{0.74,0.48,0.00}{{#1}}}
    \newcommand{\AttributeTok}[1]{\textcolor[rgb]{0.49,0.56,0.16}{{#1}}}
    \newcommand{\InformationTok}[1]{\textcolor[rgb]{0.38,0.63,0.69}{\textbf{\textit{{#1}}}}}
    \newcommand{\WarningTok}[1]{\textcolor[rgb]{0.38,0.63,0.69}{\textbf{\textit{{#1}}}}}
    
    
    % Define a nice break command that doesn't care if a line doesn't already
    % exist.
    \def\br{\hspace*{\fill} \\* }
    % Math Jax compatability definitions
    \def\gt{>}
    \def\lt{<}
    % Document parameters
    \title{Computational Linguistics and Text Analysis hw1}
    
    
    

    % Pygments definitions
    
\makeatletter
\def\PY@reset{\let\PY@it=\relax \let\PY@bf=\relax%
    \let\PY@ul=\relax \let\PY@tc=\relax%
    \let\PY@bc=\relax \let\PY@ff=\relax}
\def\PY@tok#1{\csname PY@tok@#1\endcsname}
\def\PY@toks#1+{\ifx\relax#1\empty\else%
    \PY@tok{#1}\expandafter\PY@toks\fi}
\def\PY@do#1{\PY@bc{\PY@tc{\PY@ul{%
    \PY@it{\PY@bf{\PY@ff{#1}}}}}}}
\def\PY#1#2{\PY@reset\PY@toks#1+\relax+\PY@do{#2}}

\expandafter\def\csname PY@tok@w\endcsname{\def\PY@tc##1{\textcolor[rgb]{0.73,0.73,0.73}{##1}}}
\expandafter\def\csname PY@tok@c\endcsname{\let\PY@it=\textit\def\PY@tc##1{\textcolor[rgb]{0.25,0.50,0.50}{##1}}}
\expandafter\def\csname PY@tok@cp\endcsname{\def\PY@tc##1{\textcolor[rgb]{0.74,0.48,0.00}{##1}}}
\expandafter\def\csname PY@tok@k\endcsname{\let\PY@bf=\textbf\def\PY@tc##1{\textcolor[rgb]{0.00,0.50,0.00}{##1}}}
\expandafter\def\csname PY@tok@kp\endcsname{\def\PY@tc##1{\textcolor[rgb]{0.00,0.50,0.00}{##1}}}
\expandafter\def\csname PY@tok@kt\endcsname{\def\PY@tc##1{\textcolor[rgb]{0.69,0.00,0.25}{##1}}}
\expandafter\def\csname PY@tok@o\endcsname{\def\PY@tc##1{\textcolor[rgb]{0.40,0.40,0.40}{##1}}}
\expandafter\def\csname PY@tok@ow\endcsname{\let\PY@bf=\textbf\def\PY@tc##1{\textcolor[rgb]{0.67,0.13,1.00}{##1}}}
\expandafter\def\csname PY@tok@nb\endcsname{\def\PY@tc##1{\textcolor[rgb]{0.00,0.50,0.00}{##1}}}
\expandafter\def\csname PY@tok@nf\endcsname{\def\PY@tc##1{\textcolor[rgb]{0.00,0.00,1.00}{##1}}}
\expandafter\def\csname PY@tok@nc\endcsname{\let\PY@bf=\textbf\def\PY@tc##1{\textcolor[rgb]{0.00,0.00,1.00}{##1}}}
\expandafter\def\csname PY@tok@nn\endcsname{\let\PY@bf=\textbf\def\PY@tc##1{\textcolor[rgb]{0.00,0.00,1.00}{##1}}}
\expandafter\def\csname PY@tok@ne\endcsname{\let\PY@bf=\textbf\def\PY@tc##1{\textcolor[rgb]{0.82,0.25,0.23}{##1}}}
\expandafter\def\csname PY@tok@nv\endcsname{\def\PY@tc##1{\textcolor[rgb]{0.10,0.09,0.49}{##1}}}
\expandafter\def\csname PY@tok@no\endcsname{\def\PY@tc##1{\textcolor[rgb]{0.53,0.00,0.00}{##1}}}
\expandafter\def\csname PY@tok@nl\endcsname{\def\PY@tc##1{\textcolor[rgb]{0.63,0.63,0.00}{##1}}}
\expandafter\def\csname PY@tok@ni\endcsname{\let\PY@bf=\textbf\def\PY@tc##1{\textcolor[rgb]{0.60,0.60,0.60}{##1}}}
\expandafter\def\csname PY@tok@na\endcsname{\def\PY@tc##1{\textcolor[rgb]{0.49,0.56,0.16}{##1}}}
\expandafter\def\csname PY@tok@nt\endcsname{\let\PY@bf=\textbf\def\PY@tc##1{\textcolor[rgb]{0.00,0.50,0.00}{##1}}}
\expandafter\def\csname PY@tok@nd\endcsname{\def\PY@tc##1{\textcolor[rgb]{0.67,0.13,1.00}{##1}}}
\expandafter\def\csname PY@tok@s\endcsname{\def\PY@tc##1{\textcolor[rgb]{0.73,0.13,0.13}{##1}}}
\expandafter\def\csname PY@tok@sd\endcsname{\let\PY@it=\textit\def\PY@tc##1{\textcolor[rgb]{0.73,0.13,0.13}{##1}}}
\expandafter\def\csname PY@tok@si\endcsname{\let\PY@bf=\textbf\def\PY@tc##1{\textcolor[rgb]{0.73,0.40,0.53}{##1}}}
\expandafter\def\csname PY@tok@se\endcsname{\let\PY@bf=\textbf\def\PY@tc##1{\textcolor[rgb]{0.73,0.40,0.13}{##1}}}
\expandafter\def\csname PY@tok@sr\endcsname{\def\PY@tc##1{\textcolor[rgb]{0.73,0.40,0.53}{##1}}}
\expandafter\def\csname PY@tok@ss\endcsname{\def\PY@tc##1{\textcolor[rgb]{0.10,0.09,0.49}{##1}}}
\expandafter\def\csname PY@tok@sx\endcsname{\def\PY@tc##1{\textcolor[rgb]{0.00,0.50,0.00}{##1}}}
\expandafter\def\csname PY@tok@m\endcsname{\def\PY@tc##1{\textcolor[rgb]{0.40,0.40,0.40}{##1}}}
\expandafter\def\csname PY@tok@gh\endcsname{\let\PY@bf=\textbf\def\PY@tc##1{\textcolor[rgb]{0.00,0.00,0.50}{##1}}}
\expandafter\def\csname PY@tok@gu\endcsname{\let\PY@bf=\textbf\def\PY@tc##1{\textcolor[rgb]{0.50,0.00,0.50}{##1}}}
\expandafter\def\csname PY@tok@gd\endcsname{\def\PY@tc##1{\textcolor[rgb]{0.63,0.00,0.00}{##1}}}
\expandafter\def\csname PY@tok@gi\endcsname{\def\PY@tc##1{\textcolor[rgb]{0.00,0.63,0.00}{##1}}}
\expandafter\def\csname PY@tok@gr\endcsname{\def\PY@tc##1{\textcolor[rgb]{1.00,0.00,0.00}{##1}}}
\expandafter\def\csname PY@tok@ge\endcsname{\let\PY@it=\textit}
\expandafter\def\csname PY@tok@gs\endcsname{\let\PY@bf=\textbf}
\expandafter\def\csname PY@tok@gp\endcsname{\let\PY@bf=\textbf\def\PY@tc##1{\textcolor[rgb]{0.00,0.00,0.50}{##1}}}
\expandafter\def\csname PY@tok@go\endcsname{\def\PY@tc##1{\textcolor[rgb]{0.53,0.53,0.53}{##1}}}
\expandafter\def\csname PY@tok@gt\endcsname{\def\PY@tc##1{\textcolor[rgb]{0.00,0.27,0.87}{##1}}}
\expandafter\def\csname PY@tok@err\endcsname{\def\PY@bc##1{\setlength{\fboxsep}{0pt}\fcolorbox[rgb]{1.00,0.00,0.00}{1,1,1}{\strut ##1}}}
\expandafter\def\csname PY@tok@kc\endcsname{\let\PY@bf=\textbf\def\PY@tc##1{\textcolor[rgb]{0.00,0.50,0.00}{##1}}}
\expandafter\def\csname PY@tok@kd\endcsname{\let\PY@bf=\textbf\def\PY@tc##1{\textcolor[rgb]{0.00,0.50,0.00}{##1}}}
\expandafter\def\csname PY@tok@kn\endcsname{\let\PY@bf=\textbf\def\PY@tc##1{\textcolor[rgb]{0.00,0.50,0.00}{##1}}}
\expandafter\def\csname PY@tok@kr\endcsname{\let\PY@bf=\textbf\def\PY@tc##1{\textcolor[rgb]{0.00,0.50,0.00}{##1}}}
\expandafter\def\csname PY@tok@bp\endcsname{\def\PY@tc##1{\textcolor[rgb]{0.00,0.50,0.00}{##1}}}
\expandafter\def\csname PY@tok@fm\endcsname{\def\PY@tc##1{\textcolor[rgb]{0.00,0.00,1.00}{##1}}}
\expandafter\def\csname PY@tok@vc\endcsname{\def\PY@tc##1{\textcolor[rgb]{0.10,0.09,0.49}{##1}}}
\expandafter\def\csname PY@tok@vg\endcsname{\def\PY@tc##1{\textcolor[rgb]{0.10,0.09,0.49}{##1}}}
\expandafter\def\csname PY@tok@vi\endcsname{\def\PY@tc##1{\textcolor[rgb]{0.10,0.09,0.49}{##1}}}
\expandafter\def\csname PY@tok@vm\endcsname{\def\PY@tc##1{\textcolor[rgb]{0.10,0.09,0.49}{##1}}}
\expandafter\def\csname PY@tok@sa\endcsname{\def\PY@tc##1{\textcolor[rgb]{0.73,0.13,0.13}{##1}}}
\expandafter\def\csname PY@tok@sb\endcsname{\def\PY@tc##1{\textcolor[rgb]{0.73,0.13,0.13}{##1}}}
\expandafter\def\csname PY@tok@sc\endcsname{\def\PY@tc##1{\textcolor[rgb]{0.73,0.13,0.13}{##1}}}
\expandafter\def\csname PY@tok@dl\endcsname{\def\PY@tc##1{\textcolor[rgb]{0.73,0.13,0.13}{##1}}}
\expandafter\def\csname PY@tok@s2\endcsname{\def\PY@tc##1{\textcolor[rgb]{0.73,0.13,0.13}{##1}}}
\expandafter\def\csname PY@tok@sh\endcsname{\def\PY@tc##1{\textcolor[rgb]{0.73,0.13,0.13}{##1}}}
\expandafter\def\csname PY@tok@s1\endcsname{\def\PY@tc##1{\textcolor[rgb]{0.73,0.13,0.13}{##1}}}
\expandafter\def\csname PY@tok@mb\endcsname{\def\PY@tc##1{\textcolor[rgb]{0.40,0.40,0.40}{##1}}}
\expandafter\def\csname PY@tok@mf\endcsname{\def\PY@tc##1{\textcolor[rgb]{0.40,0.40,0.40}{##1}}}
\expandafter\def\csname PY@tok@mh\endcsname{\def\PY@tc##1{\textcolor[rgb]{0.40,0.40,0.40}{##1}}}
\expandafter\def\csname PY@tok@mi\endcsname{\def\PY@tc##1{\textcolor[rgb]{0.40,0.40,0.40}{##1}}}
\expandafter\def\csname PY@tok@il\endcsname{\def\PY@tc##1{\textcolor[rgb]{0.40,0.40,0.40}{##1}}}
\expandafter\def\csname PY@tok@mo\endcsname{\def\PY@tc##1{\textcolor[rgb]{0.40,0.40,0.40}{##1}}}
\expandafter\def\csname PY@tok@ch\endcsname{\let\PY@it=\textit\def\PY@tc##1{\textcolor[rgb]{0.25,0.50,0.50}{##1}}}
\expandafter\def\csname PY@tok@cm\endcsname{\let\PY@it=\textit\def\PY@tc##1{\textcolor[rgb]{0.25,0.50,0.50}{##1}}}
\expandafter\def\csname PY@tok@cpf\endcsname{\let\PY@it=\textit\def\PY@tc##1{\textcolor[rgb]{0.25,0.50,0.50}{##1}}}
\expandafter\def\csname PY@tok@c1\endcsname{\let\PY@it=\textit\def\PY@tc##1{\textcolor[rgb]{0.25,0.50,0.50}{##1}}}
\expandafter\def\csname PY@tok@cs\endcsname{\let\PY@it=\textit\def\PY@tc##1{\textcolor[rgb]{0.25,0.50,0.50}{##1}}}

\def\PYZbs{\char`\\}
\def\PYZus{\char`\_}
\def\PYZob{\char`\{}
\def\PYZcb{\char`\}}
\def\PYZca{\char`\^}
\def\PYZam{\char`\&}
\def\PYZlt{\char`\<}
\def\PYZgt{\char`\>}
\def\PYZsh{\char`\#}
\def\PYZpc{\char`\%}
\def\PYZdl{\char`\$}
\def\PYZhy{\char`\-}
\def\PYZsq{\char`\'}
\def\PYZdq{\char`\"}
\def\PYZti{\char`\~}
% for compatibility with earlier versions
\def\PYZat{@}
\def\PYZlb{[}
\def\PYZrb{]}
\makeatother


    % Exact colors from NB
    \definecolor{incolor}{rgb}{0.0, 0.0, 0.5}
    \definecolor{outcolor}{rgb}{0.545, 0.0, 0.0}



    
    % Prevent overflowing lines due to hard-to-break entities
    \sloppy 
    % Setup hyperref package
    \hypersetup{
      breaklinks=true,  % so long urls are correctly broken across lines
      colorlinks=true,
      urlcolor=urlcolor,
      linkcolor=linkcolor,
      citecolor=citecolor,
      }
    % Slightly bigger margins than the latex defaults
    
    \geometry{verbose,tmargin=1in,bmargin=1in,lmargin=1in,rmargin=1in}
    
    

    \begin{document}


	\thispagestyle{empty}    % +1 - это титульный лист
	
	\begin{center}
		
		
		THE RUSSIAN GOVERNMENT \\
		FEDERAL STATE AUTONOMUS EDUCATIONAL INSTITUTION \\ FOR HIGHER PROFESSIONAL EDUCATION \\ NATIONAL RESEARCH UNIVERSITY \\ ''HIGHER SCHOOL OF ECONOMICS''
		
		\large
		\vspace{2 cm}
		%	\textsf{
		%	Факультет экономических наук
		%	\\ Образовательная программа «Экономика» %}           %ТЕКСТ БЕЗ ЗАСЕЧЕК
	\end{center}
	
	\vspace{2 cm}
	\begin{center}
		%	\vspace{13ex}
		\vspace{1 cm} \textbf{Компьютерная лингвистика и анализ текста} \\ \vspace{0.5 cm} Домашняя работа 1 \\ Вариант F 
		\\ Исследование качества разрешения морфологической омонимии \textit{pymorhpy2}
		
	\end{center}
	
	\vspace{2 cm}
	
	\begin{flushright}
		%	\noindent
		{  Boris Tseitlin }
		
		\vspace{1 cm}
		
		{ \textbf{"Науки о данных" } \\ 1 курс\\
			Факультет Компьютерных Наук }
	\end{flushright}
	
	\begin{center}
		\vfill
		Москва 2019
	\end{center}
	
	\newpage
	\tableofcontents
	\newpage
	
	
	\section{Введение}
	
	В данном исследовании рассматривается качество разрешения морфологической омонимии морфоанализатора русского языка \textit{pymorphy2}.
Исследование проводится с помощью размеченных текстов со снятой омонимией из корпуса \textit{OpenCorpora}.  Рассматривается омонимия по частям речи, морфологическим признакам и лемме.

	\subsection{Омонимия}

	Морфологическая омонимия – совпадение форм одного и того же слова.


	Рассмотрим для примера предложение: \begin{center}"Предметом обсуждения стали вопросы сотрудничества в международных организациях."\end{center}


	Слово "стали" является морфологическим омонимом: это может быть существительное в родительном падеже, с нормальной формой "сталь", или глаголом в множественном числе с нормальной формой "стать". Морфологически анализаторы способы с некоторой точность автоматически разрешать омонимию.
	\subsection{pymorphy2}

	\textit{pymorhpy2} это морфологический анализатор для русского языка, распространяемый по лицензии MIT. Данный морфоанализатор использует словари на основе корпуса \textit{OpenCorpora}. Он способен разрешать морфологическую омонимию через статистические методы. Для омонима морфоанализатор предлагает несколько вариантов разбора, а так же оценку вероятности каждого из них.
	\newpage
	Например:
		
\begin{lstlisting}
>>> import pymorphy2
>>> morph = pymorphy2.MorphAnalyzer()
>>> morph.parse('стали')
[Parse(word='стали', tag=OpencorporaTag('VERB,perf,intr plur,past,indc'), normal_form='стать', score=0.983766, methods_stack=((<DictionaryAnalyzer>, 'стали', 884, 4),)),
 Parse(word='стали', tag=OpencorporaTag('NOUN,inan,femn sing,gent'), normal_form='сталь', score=0.003246, methods_stack=((<DictionaryAnalyzer>, 'стали', 12, 1),)),
 Parse(word='стали', tag=OpencorporaTag('NOUN,inan,femn sing,datv'), normal_form='сталь', score=0.003246, methods_stack=((<DictionaryAnalyzer>, 'стали', 12, 2),)),
 Parse(word='стали', tag=OpencorporaTag('NOUN,inan,femn sing,loct'), normal_form='сталь', score=0.003246, methods_stack=((<DictionaryAnalyzer>, 'стали', 12, 5),)),
 Parse(word='стали', tag=OpencorporaTag('NOUN,inan,femn plur,nomn'), normal_form='сталь', score=0.003246, methods_stack=((<DictionaryAnalyzer>, 'стали', 12, 6),)),
 Parse(word='стали', tag=OpencorporaTag('NOUN,inan,femn plur,accs'), normal_form='сталь', score=0.003246, methods_stack=((<DictionaryAnalyzer>, 'стали', 12, 9),))]
\end{lstlisting}
	

	\subsection{OpenCorpora}
	\textit{OpenCorpora} это открытый корпус для русского языка. Он содержит, на данный момент, 108960 размеченных предложений, состоящих из 1966780. 
	На следующем изображении представлены обозначения граммем в корпусе. Они совпадают с обозначениями, которые использует \textit{pymorphy}.
	    \begin{center}
		\adjustimage{max size={0.75\linewidth}{0.6\paperheight}}{img/1_3_grammemes.png}
	    \end{center}

	\subsection{Методология}
	Перед нами стояла задача исследовать качество разрешения омонимии. Был взят подкорпус \textit{OpenCorpora} со снятой омонимией, состоящий из 68819 токенов. Для каждого токена мы сравнивали, как \textit{pymorphy2} разрешает омонимию с тем, как она была разрешена в подкорпусе.


	Исследование производилось следующим образом:
	\begin{enumerate}
  		\item Каждый токен подкорпуса разбирался с помощью \textit{pymorhpy2}.
  		\item Если \textit{pymorphy2} предлагал несколько возможных разборов токена, то токен считался омонимом.
		\item Вариант разбора с наибольшим \textit{score} считался результатом разрешения омонимии от \textit{pymorhpy2}.
		\item Омонимы, результаты разрешения омонимии и истинные значения по корпусу были сохранены в таблицу.
		\item Таблица была проанализирована для определения качества разрешения омонимии.
	\end{enumerate}

	\section{Исследование}

	\subsection{Формат данных}
	Данные для анализа имели следующий вид:

	\begin{center}
		\fbox{\adjustimage{max size={1\linewidth}{0.9\paperheight}}{img/2_1_table.png}}
    \end{center}
	
	\noindent
	\textit{corpora\_lemma} - лемма по корпусу \\
	\textit{morph\_lemma} - лемма согласно морфоанализатору \\
	\textit{corpora\_pos} - часть речи по корпусу \\
	\textit{morph\_pos} - часть речи согласно морфоанализатору \\
	\textit{corpora\_g} - набор тегов по корпусу \\
	\textit{morph\_g} - набор тегов согласно морфоанализатору \\
	\noindent
	Всего было проанализировано \textbf{6644} омонима.

	\subsection{Разрешение омонимии по частям речи}
	Когда слово может быть трактовано как разные части речи, мы считаем, что это омонимия по части речи (частерченая омонимия). Например слово "стали" является омонимом по части речи.

	Для изучения разрешения частеречной омонимии мы составили тепловую карту:
	\begin{center}
		\fbox{\adjustimage{max size={1\linewidth}{0.9\paperheight}}{img/2_2_heatmap.png}}
    \end{center}
	
	По вертикали на ней отложены части речи из корпуса, по горизонтали - части речи, распознанные морфоанализатором. Значения клеток на пересечении показывают число токенов. Например, карта показывает, что \textbf{3663} существительных из корпуса были корректно распознаны анализатором как существительные, но \textbf{53} существительных были распознаны как полные прилагательные.
	По тепловой карте сразу видно, что большинство токенов это существительные, прилагательные и глаголы. Доля остальных частей в корпусе речи очень мала.

	\begin{table}[h]
		\begin{center}
		\begin{tabular}{lrr}
				\toprule
				{} &  count &  errors \\
				corpora\_pos &        &         \\
				\midrule
				ADJF        &   1661 &     255 \\
				ADJS        &     23 &       9 \\
				ADVB        &     36 &      31 \\
				COMP        &     42 &      10 \\
				CONJ        &     58 &      58 \\
				GRND        &      1 &       0 \\
				INFN        &      1 &       1 \\
				INTJ        &     32 &      32 \\
				NOUN        &   3808 &     145 \\
				NPRO        &    155 &      66 \\
				NUMR        &     16 &       1 \\
				PRCL        &    106 &     106 \\
				PREP        &     29 &      29 \\
				PRTF        &    158 &       9 \\
				PRTS        &      5 &       0 \\
				ROMN        &     11 &      11 \\
				VERB        &    502 &      79 \\
				\bottomrule
				\end{tabular}
		\caption{Количество омонимов и ошибок по частям речи}
		\end{center}
	\end{table}

	Наиболее често прилагательные интерпретируются как существительные (155), или как местоимения-существительные (81).

	\begin{table}[h]
		\begin{center}
		\begin{tabular}{lllr}
		\toprule
		{} & corpora\_pos & morph\_pos &  count \\
		\midrule
		4  &        ADJF &      NOUN &    155 \\
		5  &        ADJF &      NPRO &     81 \\
		33 &        NOUN &      ADJF &     53 \\
		53 &        PRCL &      ADVB &     51 \\
		34 &        NOUN &      ADJS &     35 \\
		46 &        NPRO &      CONJ &     32 \\
		60 &        PREP &      ADVB &     28 \\
		70 &        VERB &      NOUN &     27 \\
		67 &        VERB &      ADJS &     27 \\
		54 &        PRCL &      CONJ &     26 \\
		\bottomrule
		\end{tabular}
		\end{center}
		\caption{10 наиболее частых ошибок при разрешении частеречной омонимии}
	\end{table}

	
	Другой результат получается если отнормировать количество ошибок распознавания каждой части речи на общее количество токенов этой части речи.
	\begin{table}[h!]
	\begin{center}
	\begin{tabular}{lllrrr}
	\toprule
	{} & corpora\_pos & morph\_pos &  count &  total\_count &  rel\_count \\
	\midrule
	26 &        INFN &      VERB &      1 &            1 &   1.000000 \\
	66 &        ROMN &      LATN &     11 &           11 &   1.000000 \\
	60 &        PREP &      ADVB &     28 &           29 &   0.965517 \\
	14 &        ADVB &      NOUN &     18 &           36 &   0.500000 \\
	53 &        PRCL &      ADVB &     51 &          106 &   0.481132 \\
	32 &        INTJ &      VERB &     14 &           32 &   0.437500 \\
	10 &        ADJS &      NOUN &      9 &           23 &   0.391304 \\
	24 &        CONJ &      VERB &     20 &           58 &   0.344828 \\
	20 &        CONJ &      ADVB &     17 &           58 &   0.293103 \\
	29 &        INTJ &      PRCL &      9 &           32 &   0.281250 \\
	\bottomrule
	\end{tabular}
	\caption{10 наиболее относительно частых ошибок относительно общего числа токенов для каждой части речи}

	\end{center}
	\end{table}

	Некоторые полученные пары нерепрезентативны из-за малого общего количества токенов (например INFN - VERB). Но можно заметить, что относительно общего числа токенов, морфоанализатор часто принимает наречия за сщуествительные (ADVB - NOUN). Так же заметно, что 28 предлогов-омонимов из 29 были распознаные как наречия.
	

	Пример предложения с такой ошибкой:
	\begin{center}
	\textit{Политическая борьба \textbf{вокруг} личности Андерсон.}
	\end{center}
	В этом предложении слово "вокруг"" является предлогом, но \textit{pymorphy2} распознаёт его как наречие.

	В заключение мы приводим качества распознавания каждой части речи.
 	
	\begin{table}[h!]
		\begin{center}
			\begin{tabular}{lr}
			\toprule
			{} &  correct\_rate \\
			corpora\_pos &               \\
			\midrule
			PRTS        &         100.0 \\
			GRND        &         100.0 \\
			NOUN        &          96.2 \\
			PRTF        &          94.3 \\
			NUMR        &          93.8 \\
			ADJF        &          84.6 \\
			VERB        &          84.3 \\
			COMP        &          76.2 \\
			ADJS        &          60.9 \\
			NPRO        &          57.4 \\
			ADVB        &          13.9 \\
			INFN        &           0.0 \\
			INTJ        &           0.0 \\
			ROMN        &           0.0 \\
			PRCL        &           0.0 \\
			PREP        &           0.0 \\
			CONJ        &           0.0 \\
			\bottomrule
			\end{tabular}
			\caption{Качество распознавания каждой части речи в процентах}
		\end{center}
	\end{table}
	
	Таким образом \textit{pymorphy2} с точностью близкой к 90\% разрешает омонимию по части речи для большинства слов - существительных, прилагательных, глаголов - но часто ошибается в необычных случаях, включающих в себя междометия, частицы, союзы, предлоги, наречия. Размеченных омонимов корпуса \textit{OpenCorpora} слишком мало, чтобы подробнее исследовать данные редкие случаи.

	\subsection{Разрешение омонимии по морфологическим признакам}

	Когда разборы по корпусу и по \textit{pymorphy2} имеют одинаковые части речи, но различные морфологические признаки, мы считаем, что это омонимия по морфологическим признакам. Например слово "стали" является омонимом по части речи.
	Например:


	\begin{center}
		\textit{Анна Андерсон в \textbf{культуре}}
	\end{center}

	Для слова "культуре" \textit{OpenCorpora} предлагает следующий набор тегов: \textit{NOUN, loct, femn, sing, inan}. \textit{pymorphy2} предлагает: NOUN, datv, femn, sing, inan. Таким образом, разборы разлчиаются падежом: морфоанализатор "считает", что падеж слова дательный, но на самом деле он предложный.

	Мы вычислили долю верно разрешаемых наборов тегов для каждой части речи. 
	Для многих чачстей речи, включая существительные, количество точно разрешенных наборов тегов очень мало. 


	\begin{table}[h!]
		\begin{center}
		\begin{tabular}{llrrr}
			\toprule
			{} & corpora\_pos &  count &  correct tokens &  correct fraction \\
			\midrule
			0  &        NOUN &   3663 &          367.0 &          0.100191 \\
			1  &        ADJF &   1406 &           69.0 &          0.049075 \\
			2  &        VERB &    423 &          423.0 &          1.000000 \\
			3  &        PRTF &    149 &          102.0 &          0.684564 \\
			4  &        NPRO &     89 &            0.0 &          0.000000 \\
			5  &        COMP &     32 &           31.0 &          0.968750 \\
			6  &        NUMR &     15 &            4.0 &          0.266667 \\
			7  &        ADJS &     14 &           14.0 &          1.000000 \\
			8  &        PRTS &      5 &            5.0 &          1.000000 \\
			9  &        ADVB &      5 &            0.0 &          0.000000 \\
			10 &        GRND &      1 &            1.0 &          1.000000 \\
			\bottomrule
			\end{tabular}

			\caption{Доля верно разрешаемых наборов тегов для каждой части речи}
		\end{center}
	\end{table}

	Насколько сильно отличаются наборы тегов по корпусу и по морфоанализатору?

	\begin{table}[h!]
		\begin{center}
			\begin{tabular}{llr}
		\toprule
		{} & corpora\_pos &  median tag difference length \\
		\midrule
		0  &        ADJF &                             1 \\
		1  &        ADJS &                             0 \\
		2  &        ADVB &                             1 \\
		3  &        COMP &                             0 \\
		4  &        GRND &                             0 \\
		5  &        NOUN &                             1 \\
		6  &        NPRO &                             1 \\
		7  &        NUMR &                             1 \\
		8  &        PRTF &                             0 \\
		9  &        PRTS &                             0 \\
		10 &        VERB &                             0 \\
		\bottomrule
		\end{tabular}
		\caption{Медиана количества различающихся тегов для каждой части речи.}
		\end{center}
	\end{table}

	\begin{center}
		\fbox{\adjustimage{max size={1\linewidth}{0.9\paperheight}}{img/2_3_boxplot.png}}
    \end{center}

	Оказалось, что чаще всего наборы тегов от морфоанализатора и по корпусу отличаются на один тег.

	Мы проверили, какая доля наборов тегов отличаются не более, чем на один тег. 

	\begin{table}[h!]
	\begin{center}
		\begin{tabular}{llrr}
		\toprule
		{} & corpora\_pos &  count &  (mostly) correct fraction \\
		\midrule
		0  &        NOUN &   2164 &                   0.590773 \\
		1  &        ADJF &    925 &                   0.657895 \\
		2  &        VERB &    423 &                   1.000000 \\
		3  &        PRTF &    140 &                   0.939597 \\
		4  &        NPRO &     82 &                   0.921348 \\
		5  &        COMP &     32 &                   1.000000 \\
		6  &        ADJS &     14 &                   0.933333 \\
		7  &        NUMR &     12 &                   0.857143 \\
		8  &        PRTS &      5 &                   1.000000 \\
		9  &        ADVB &      5 &                   1.000000 \\
		10 &        GRND &      1 &                   1.000000 \\
		\bottomrule
		\end{tabular}
		\caption{Доля наборов тегов, отличающихся не более, чем на один тег.}
		\end{center}
	\end{table}

	Можно сделать вывод, что для всех частей речи \textit{pymorphy2} разрешает набор тегов с не более чем одной ошибкой с точностью более 59\%.
	\begin{table}[h!]
	\begin{center}
	\begin{tabular}{lr}
		\toprule
		grammemes diff &  count \\
		\midrule
		\{'plur', 'nomn'\} &     688 \\
		\{'accs'\}         &     639 \\
		\{'loct'\}         &     542 \\
		\{'nomn'\}         &     469 \\
		\{'gent'\}         &     293 \\
		\{'plur', 'accs'\} &     281 \\
		\{'datv'\}         &     210 \\
		\{'inan', 'accs'\} &     208 \\
		\{'neut'\}         &     164 \\
		\{'ablt'\}         &     131 \\
		\bottomrule
		\end{tabular}
		\caption{10 наиболее часто встречающихся ошибок при распознавании}
		\end{center}
	\end{table}


	\begin{table}[h!]
		\begin{center}
			\begin{tabular}{lllr}
			\toprule
			{} & corpora\_pos &            grammemes diff &  count \\
			\midrule
			185 &        NOUN &  \{'plur', 'nomn'\} &    688 \\
			112 &        NOUN &          \{'accs'\} &    606 \\
			165 &        NOUN &          \{'nomn'\} &    412 \\
			168 &        NOUN &  \{'plur', 'accs'\} &    281 \\
			143 &        NOUN &          \{'loct'\} &    265 \\
			33  &        ADJF &          \{'loct'\} &    249 \\
			32  &        ADJF &  \{'inan', 'accs'\} &    201 \\
			138 &        NOUN &          \{'gent'\} &    184 \\
			122 &        NOUN &          \{'datv'\} &    155 \\
			45  &        ADJF &          \{'neut'\} &    153 \\
			\bottomrule
			\end{tabular}
			\caption{10 наиболее часто встречающихся ошибок при распознавании по частям речи}
		\end{center}
	\end{table}

	\begin{table}[h!]
		\begin{center}
		\begin{tabular}{lllr}
			\toprule
			{} & corpora\_pos &            grammemes diff &  count \\
			\midrule
			6 &        NOUN &  \{'plur', 'nomn'\} &    688 \\
			5 &        ADJF &          \{'loct'\} &    249 \\
			4 &        NPRO &          \{'gent'\} &     39 \\
			3 &        PRTF &          \{'loct'\} &     12 \\
			1 &        ADVB &          \{'Prdx'\} &      5 \\
			2 &        NUMR &          \{'loct'\} &      5 \\
			0 &        COMP &          \{'Qual'\} &      1 \\
			\bottomrule
			\end{tabular}
			\caption{Наиболее часто встречающиеся ошибки по частям речи}
		\end{center}
	\end{table}

	Наибольшие проблемы возникают при разрешении множественного числа существительных, именительного падежа существительных, предложного падежа прилагательных, родительного падежа местоимений.  

	\subsection{Разрешение омонимии по лемме}

	В случае, если лемма слова по морфоанализатору и по корпусу различается, но часть речи и морфологические признаки одинаковы, мы называем это омонимией по лемме.
	Например:
	\begin{center}
		\textit{А шанса на счастье \textbf{может} уже и не быть.}
	\end{center}
	Слово "может" может иметь лемму "мочь" или "могу", но в любом случае это глагол.

	Омонимия по лемме это редкое явление.
	Всего было найдено \textbf{656} таких омонимов.

	\begin{center}
		\fbox{\adjustimage{max size={1\linewidth}{0.9\paperheight}}{img/2_4_bar.png}}
    \end{center}

    Было обнаружено, что большинство омонимов по лемме являются глаголами.

    \begin{center}
		\fbox{\adjustimage{max size={1\linewidth}{0.9\paperheight}}{img/2_4_bar_2.png}}
    \end{center}


	\section{Заключение}

	В ходе исследования мы оценили качество разрешения морфологической омонимии морфоанализатором \textit{pymorphy2}. Мы рассмотрели омонимию по части речи, по морфологическим признакам и по лемме. Можно сделать вывод, что в большинстве случаев точность разрешения \textit{pymorphy2} достаточно высока. Возникают проблемы при работе со сложными и редкими случаями, например при различении наречий и существительных. Полученные данные могут быть использованы для выявления "проблемных" ситуаций и улучшения качества разрешенияы. 
 
\end{document}